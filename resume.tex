%%%%%%%%%%%%%%%%%%%%%%%%%%%%%%%%%%%%%%%%%
% Medium Length Professional CV
% LaTeX Template
% Version 2.0 (8/5/13)
%
% This template has been downloaded from:
% http://www.LaTeXTemplates.com
%
% Original author:
% Trey Hunner (http://www.treyhunner.com/)
%
% Important note:
% This template requires the resume.cls file to be in the same directory as the
% .tex file. The resume.cls file provides the resume style used for structuring the
% document.
%
%%%%%%%%%%%%%%%%%%%%%%%%%%%%%%%%%%%%%%%%%

%----------------------------------------------------------------------------------------
%	PACKAGES AND OTHER DOCUMENT CONFIGURATIONS
%----------------------------------------------------------------------------------------

\documentclass{resume} % Use the custom resume.cls style

\usepackage{enumitem}
\setlist{nosep}

\usepackage{xcolor}

\usepackage[left=0.4 in,top=0.4in,right=0.4 in,bottom=0.4in]{geometry} % Document margins
\newcommand{\tab}[1]{\hspace{.2667\textwidth}\rlap{#1}}
\newcommand{\itab}[1]{\hspace{0em}\rlap{#1}}
\name{Ayberk Yaraneri} % Your name
%\address{'Sampatham' PO Edakkad, Westhill,Kozhikode,Kerala,India} % Your secondary addess (optional)
\address{217 390 1132 \\ ayberkyaraneri@gmail.com \\ github.com/ayberky}  % Your phone number and email

\textbf{NON-U.S. Citizen}

\begin{document}

    %----------------------------------------------------------------------------------------
    %	TECHNICAL COMPETENCIES
    %----------------------------------------------------------------------------------------

    \begin{rSection} {Technical Competencies}

        \begin{tabular}{ @{} >{\bfseries}l @{\hspace{2ex}} l }

        Languages: & (Proficient) C/C++, MIPS Assembly, Verilog, Python, MATLAB, (Familiar) Java, LaTeX\\
        Tools: & Git, Vim, Conda, GDB, TensorFlow, Keras, OpenVino, Gazebo, Robot Operating System \\
        Relevant Skills: & Linux CLI, Embedded Systems, PID Control, I$^2$C, SPI, UART, Object Oriented Programming \\ & Data Structures, Machine Learning, Computer Vision, Board/Chip Bringup, Datasheet Sifting \\
		Relevant Coursework: & CS225 Data Structures, CS233 Computer Architecture, CS357 Numerical Methods \\
        % CAD: & Siemens NX, Solidworks \\
        % Other: & 3D Printing, CNC Laser Cutter

        \end{tabular}

    \end{rSection}

    % \smallskip

    %----------------------------------------------------------------------------------------
    %	EXPERIENCE
    %----------------------------------------------------------------------------------------

    \begin{rSection}{Experience}

        \begin{rSubsection}{Illinois Applied Research Institute} {February 2019 - September 2019} {Full Time Robotics Developer}

            \item Assisted in the development of autonomous multirotor UAVs intended for a simulated reconnaissance mission utilizing convolutional neural networks for detection and tracking of ground ajents.
            \item Conducted transfer learning on various object detection networks such as Faster R-CNN, SSD, and YOLO.
            \item Optimized trained neural networks using the OpenVino toolkit to run on an Intel Movidius Neural Computer Stick for accelerated on-oard inference.
            \item Configured Raspberry Pi computers to work with the Movidius NCS and transmit observations as Mavlink messages through the Pixhawk flight controller's telemetry connection.
            \item Wrote code to automate data collection and labelling which expedited the training process.
            % \item Worked with Ardupilot-SITL and Gazebo as a physics simulator.

        \end{rSubsection}

    \end{rSection}

    %----------------------------------------------------------------------------------------
    %	Leadership and Activities
    %----------------------------------------------------------------------------------------

    \begin{rSection} {leadership and Projects}

        \begin{rSubsection} {NASA Student Launch Rocketry Competition} {September 2019 - April 2020} {Chief Engineer of Payload}

            \item Collaborated with Project Manager in leading the development of an air deployed autonomous quadrotor intended to deploy from a rocket during descent. Aircraft is tasked to execute a simulated ice sample retrieval mission.
            \item Employed the Navio2 platform for hard real time IO coupled with a Raspberry Pi running the open source Ardupilot flight stack for autonomous guidance navigation and control.
            \item Implemented computer vision algorithms in C++ to run on the Raspberry Pi which to detect and guide the UAV's descent towards the ice retrieval site.
            % \item Determined requirements and constraints of said CV algorithms and evaluated their implementations based on factors such as computational overhead, reliability, accuracy, and risk of interfering with the Ardupilot flight stack.
            \item Applied software-in-the-loop methods using Ardupilot-SITL and Gazebo as a physics engine to thoroughly test corner cases and validate software reliability.
            % \item Formulating a development, manufacturing, and testing schedule for software and structural components of the aircraft.


        \end{rSubsection}

        %--------------------------------------------------------------------------------------------


        \begin{rSubsection} {NASA Midwest High Power Rocketry Competition} {September 2018 - September 2019} {Avionics Sub-Team Lead}

            \item Lead an all-freshman team in developing an avionics package tasked to collect performance data of a supersonic high powered rocket using a variety of sensors.
            \item Embedded a Raspberry Pi Zero as the primary flight computer which utilized I$^2$C, SPI, and UART communication protocols to acquire data from on-board sensors.
            \item Assigned and oversaw the development of flight software written in Python for all sub systems.
            \item Successfully incorporated Git as a version control and collaboration tool which significantly enhanced the team's work flow.
            \item Coordinated the development and assembly of a printed circuit board allowing for a more streamlined design.
            \item Placed 2$^{nd}$ overall in competition completing five flights, two of which were supersonic.

        \end{rSubsection}

        \smallskip

        %--------------------------------------------------------------------------------------------

        \begin{rSubsection} {Spaceport America Cup Rocketry Competition} {September 2018 - June 2019} {Avionics Team Member}

            \item Developed an on-board flight computer to actuate external control surfaces for roll control and active drag manipulation.
            \item Embedded an Atmega328P microcontroller and wrote flight software implementing a closed loop PID controller.
            \item Designed and assembled a printed circuit board that served as the primary structural member of the flight computer.
            \item Assisted in the development of Wi-Fi enbaled solid state switches using ESP8266 microcontrollers to wirelessly toggle power to onboard systems.

        \end{rSubsection}

    \end{rSection}

    %----------------------------------------------------------------------------------------
    %	EDUCATION SECTION
    %----------------------------------------------------------------------------------------

    \begin{rSection}{Education}

        {\textbf{University of Illinois at Urbana Champaign}} \hfill {\em 2018 - 2022}\\
        Bachelor of Science in Aerospace Engineering \hfill {\em Technical GPA : 3.71/4.00} \\
        Minor in Computer Science \hfill {\em Overall GPA : 3.50/4.00}


    \end{rSection}

\end{document}
