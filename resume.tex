%%%%%%%%%%%%%%%%%%%%%%%%%%%%%%%%%%%%%%%%%
% Medium Length Professional CV
% LaTeX Template
% Version 2.0 (8/5/13)
%
% This template has been downloaded from:
% http://www.LaTeXTemplates.com
%
% Original author:
% Trey Hunner (http://www.treyhunner.com/)
%
% Important note:
% This template requires the resume.cls file to be in the same directory as the
% .tex file. The resume.cls file provides the resume style used for structuring the
% document.
%
%%%%%%%%%%%%%%%%%%%%%%%%%%%%%%%%%%%%%%%%%

%----------------------------------------------------------------------------------------
%	PACKAGES AND OTHER DOCUMENT CONFIGURATIONS
%----------------------------------------------------------------------------------------

\documentclass{resume} % Use the custom resume.cls style

\usepackage{enumitem}
\setlist{nosep}

\usepackage{xcolor}

\usepackage[left=0.4 in,top=0.4in,right=0.4 in,bottom=0.4in]{geometry} % Document margins
\newcommand{\tab}[1]{\hspace{.2667\textwidth}\rlap{#1}}
\newcommand{\itab}[1]{\hspace{0em}\rlap{#1}}
\name{Ayberk Yaraneri} % Your name
%\address{'Sampatham' PO Edakkad, Westhill,Kozhikode,Kerala,India} % Your secondary addess (optional)
\address{217 390 1132 \\ ayberkyaraneri@gmail.com \\ github.com/ayberky}  % Your phone number and email

\textbf{NON-U.S. Citizen}

\begin{document}

    %----------------------------------------------------------------------------------------
    %	TECHNICAL COMPETENCIES
    %----------------------------------------------------------------------------------------

    \begin{rSection} {Technical Competencies}

        \begin{tabular}{ @{} >{\bfseries}l @{\hspace{2ex}} l }

        Languages: & C/C++, Python, MIPS Assembly, Verilog, MATLAB, Java, LaTeX \\
        Tools: & Git, Vim, GDB, Crostool-NG, Conda, TensorFlow, Keras, ROS, Oscilloscope/Logic Analyzer \\
        Relevant Skills: & Linux CLI, Embedded Systems, Peripheral Drivers/Firmware, I$^2$C, SPI, UART, JTAG, \\ & Data Structures, Machine Learning, Computer Vision, Board/Chip Bringup \\
		Relevant Coursework: & \textbf{ECE391} Computer System Programming, \textbf{CS433} Computer System Organization,\\ & \textbf{CS233} Computer Architecture, \textbf{CS225} Data Structures, \textbf{CS357} Numerical Methods \\ & \textbf{AE353} Aerospace Control Systems, \textbf{AE199} Aerospace Computation

        \end{tabular}

    \end{rSection}

    % \smallskip

    %----------------------------------------------------------------------------------------
    %	EXPERIENCE
    %----------------------------------------------------------------------------------------

    \begin{rSection}{Experience}

		\begin{rSubsection}{Raptee Energy} {May 2020 - August 2020} {Embedded Systems Engineering Internship}

			\item Lead a group of interns in the early development of a vehicle-control-unit for electric motorcycles.
			\item Devised various homogeneous and heterogeneous system architectures and throughly assessed trade-offs between designs.
			\item Developed a FIFO buffer data structure implementing mutual exclusion principles allowing for non-blocking parallel execution.
			\item Wrote driver firmware in C for the SPI, ADC, and FlexTimer peripherals of an NXP MK20DX256 microcontroller.
			\item Created a benchmarking tool to evaluate and compare the multi-threaded performance of hardware options.
			\item Configured cross-compiler toolchains for the prototype hardware and made them available to the team.

		\end{rSubsection}

        \begin{rSubsection}{Illinois Applied Research Institute} {February 2019 - September 2019} {Full Time Robotics Developer}

            \item Assisted in the development of UAVs utilizing convolutional neural networks for detection and tracking of ground robots.
            \item Conducted transfer learning on various object detection networks such as Faster R-CNN, SSD, and YOLO.
            \item Optimized trained models using the OpenVino toolkit to run on an Intel Movidius NCS for accelerated on-board inference.
            \item Wrote code to automate data collection and labelling which expedited the training process.

        \end{rSubsection}

    \end{rSection}

    %----------------------------------------------------------------------------------------
    %	Leadership and Activities
    %----------------------------------------------------------------------------------------

    \begin{rSection} {leadership and Projects}

        \begin{rSubsection} {NASA Student Launch Rocketry Competition} {September 2019 - April 2020} {Chief Engineer of Payload}

            \item Collaborated with Project Manager in leading the development of an autonomous UAV that is deploy from a rocket in-flight.
            \item Employed the Navio2 platform for hard real time IO coupled with a Raspberry Pi running Ardupilot for autonomy.
            \item Implemented computer vision algorithms in C++ to run on the Raspberry Pi for close-quarter guidance.
            \item Applied software-in-the-loop methods using Gazebo as a physics engine to test corner cases and validate software reliability.


        \end{rSubsection}

        %--------------------------------------------------------------------------------------------


        \begin{rSubsection} {NASA Midwest High Power Rocketry Competition} {September 2018 - September 2019} {Avionics Sub-Team Lead}

            \item Lead an all-freshman team in developing an avionics package to collect flight data of a supersonic rocket using various sensors.
            \item Embedded a Raspberry Pi Zero as the primary flight computer which utilized I$^2$C, SPI, and UART communication protocols.
            \item Assigned and oversaw the development of flight software written in Python for all sub systems.
            \item Successfully incorporated Git as a version control and collaboration tool which significantly enhanced the team's work flow.
            \item Coordinated the development and assembly of a printed circuit board allowing for a more streamlined design.
            \item Placed 2$^{nd}$ overall in competition completing five flights, two of which were supersonic.

        \end{rSubsection}

        \smallskip

        %--------------------------------------------------------------------------------------------

        \begin{rSubsection} {Spaceport America Cup Rocketry Competition} {September 2018 - June 2019} {Avionics Team Member}

            \item Developed an on-board flight computer to actuate external control surfaces for roll control and active drag manipulation.
            \item Embedded an Atmega328P microcontroller and wrote flight software implementing a closed loop PID controller.
            \item Designed and assembled a printed circuit board that served as the primary structural member of the flight computer.
            \item Assisted in the development of Wi-Fi enbaled solid state switches using ESP8266 microcontrollers.
        \end{rSubsection}

    \end{rSection}

    %----------------------------------------------------------------------------------------
    %	EDUCATION SECTION
    %----------------------------------------------------------------------------------------

    \begin{rSection}{Education}

        {\textbf{University of Illinois at Urbana Champaign}} \hfill {\em 2018 - 2022}\\
        Bachelor of Science in Aerospace Engineering \hfill {\em Technical GPA : 3.77/4.00} \\
        Minor in Computer Science \hfill {\em Overall GPA : 3.55/4.00}


    \end{rSection}

\end{document}
